\documentclass{article}
\usepackage[utf8]{inputenc}
\usepackage[margin = 0.8in]{geometry}
\usepackage{graphicx}
\usepackage{amsmath, amssymb}
\usepackage{subcaption}
\usepackage{multirow}
\usepackage{mathtools}
\usepackage{float}


\title{RBE502 - Homework Set 3}
\author{Keith Chester}
\date{Due date: September 15, 2021}

\begin{document}
\maketitle

\section*{Problem Statement}
In this problem, we are given a series of equations for a rotating rigid bod spacecraft, seen below.

\[J_1 \dot{\omega}_1 = (J_2 - J_3)\omega_2 \omega_3 + u_1\]
\[J_2 \dot{\omega}_2 = (J_3 - J_1)\omega_3 \omega_1 + u_2\]
\[J_3 \dot{\omega}_3 = (J_1 - J_2)\omega_1 \omega_2 + u_3\]

In these equations, $\omega_1$, $\omega_2$, and $\omega_3$ are components of the angular velocity ${\boldmath{\omega}}$ along the principal axes. $u_1$, $u_2$, and $u_3$ are torque inputs along these same axes. $J_1$, $J_2$, and $J_3$ are the principal mass moments of inertia.

\section*{Part A}

In this problem, we use Lyapunov's theory of stability to determine that the system is stable for $\boldmath{\omega}=0$, and if so is it asymptotically stable?

First, we look again at the original equations for the system presented to us, but we replace each variable for the torque input of $\boldmath{u} = 0$.

\[J_1 \dot{\omega}_1 = (J_2 - J_3)\omega_2 \omega_3 \]
\[J_2 \dot{\omega}_2 = (J_3 - J_1)\omega_3 \omega_1 \]
\[J_3 \dot{\omega}_3 = (J_1 - J_2)\omega_1 \omega_2 \]

We then look towards setting an $x_1$ and $x_2$ for the system to observe its behaviour, as we did in the prior homework. However, we have three equations for the entire system instead of a singular one, one for each principal axes. Instead of utilizing variables $x_1$ and $x_2$ to replace a singular variable, we will make use of $x_{1_i}$ and $x_{2_i}$ for $\omega_i$.

\[ \boldmath{x} = \begin{bmatrix}
    x_1 \\ x_2
\end{bmatrix} = \begin{bmatrix}
    x_{1_1} \\
    x_{1_2} \\ 
    x_{1_3} \\
    \\
    x_{2_1} \\
    x_{2_2} \\
    x_{2_3} 
\end{bmatrix} =
\begin{bmatrix}
    \omega_1 \\
    \omega_2 \\
    \omega_3 \\
    \\ 
    \frac{(J_2 - J_3)\omega_2 \omega_3}{J_1} \\
    \frac{(J_3 - J_1)\omega_3 \omega_1}{J_2} \\
    \frac{(J_1 - J_2)\omega_1 \omega_2}{J_3}
\end{bmatrix}\]

We can use the Lagrange equation as our $V(\omega)$ to look at our system, which involves looking at the sum of kinetic and potential energy in our system; but we must note that since this is a satellite in space, there exists no potential energy. So instead of  $V(x_1) = \sum{\frac{1}{2}m_iv_i^2+m_igh_i}$, we will just use $V(x_1) = \sum{\frac{1}{2}m_iv_i^2}$. Here, $J_i$ as our inertial moment of inertia shall be our $m_i$ and our $v_i$ is the angular velocity $x_1$.

\[ V(x_1) = \sum{\frac{1}{2} J_i x_{1_i}^2} \]
\[ V(x_1) = \frac{1}{2} J_1 x_{1_1}^2 + \frac{1}{2} J_2 x_{1_2}^2 + \frac{1}{2} J_3 x_{1_3}^2 \]

We then want to explore the value of $\dot{V}(x_1)$, which we do via deriving with the chain rule. Specifically, $\frac{1}{2}J_i x_1^2$ with the chain rule ($\frac{d}{du}J_i u = J_i u \dot{u})$, is thus $\frac{1}{2} J_i 2 x_1 \dot{x}_1 = J_i x_1 \dot{x}_1$, which since $\dot{x}_1 = x_2$, it now becomes $J_i x_1 x_2$. So our full $\dot{V}(x_1)$ becomes...

\[ \dot{V}(x_1) = J_1 x_{1_1} x_{2_1} + J_2 x_{1_2} x_{2_2} + J_3 x_{1_3} x_{2_3}\]

Now we subsibtute the values of each $x_{1_i}$ and $x_{2_i}$.

\[ \dot{V}(\omega) = J_1 \omega_1 \frac{(J_2 - J_3)\omega_2 \omega_3}{J_1} + J_2 \omega_2 \frac{(J_3 - J_1)\omega_3 \omega_1}{J_2} + J_3 \omega_3 \frac{(J_1 - J_2)\omega_1 \omega_2}{J_3}\]

\[ \dot{V}(\omega) = (J_2-J_3)\omega_1 \omega_2 \omega_3 + (J_3-J_1) \omega_1 \omega_2 \omega_3 + (J_1-J_2) \omega_1 \omega_2 \omega_3\]

To determine if this is a Lyapunov function, three conditions must be met:
\begin{enumerate}
    \item $V(\omega^*)=0$
    \item $V(\omega) > 0 \quad \forall \omega \epsilon D \backslash \{\omega^*\}$
    \item $\dot{V}(\omega) \leq 0$ in $D$
\end{enumerate}
Additionally, the system is asymptotically stable if $\dot{V}(\omega) < 0 \quad \forall \omega \epsilon D \backslash \{\omega^*\}$. For our system, we are looking at $\omega^* = 0$.

\[ V(\omega^*) = V(0) = \frac{1}{2} J_1 0^2 + \frac{1}{2} J_2 0^2 + \frac{1}{2} J_3 0^2 = 0 \]

We see that $V(\omega^*) = 0$, thus covering the first condition. Since each $\omega$ value is squared, any value used for $\omega$ will result in a positive value for $V(\omega)$. This covers our second condition. If we look at the derivative of $V(\omega)$, $\dot{V}(\omega)$:

\[ \dot{V}(\omega) = (J_2-J_3)\omega_1 \omega_2 \omega_3 + (J_3-J_1) \omega_1 \omega_2 \omega_3 + (J_1-J_2) \omega_1 \omega_2 \omega_3 \]
\[
    \dot{V}(\omega) = J_2 \omega_1 \omega_2 \omega_3 - J_3 \omega_1 \omega_2 \omega_3 + J_3 \omega_1 \omega_2 \omega_3 - J_1 \omega_1 \omega_2 \omega_3 + J_1 \omega_1 \omega_2 \omega_3 - J_2 \omega_1 \omega_2 \omega_3
\]
\[ \dot{V}(\omega) = 0\]

Thus, we can see that this is a stable Lyapunov function as all aforementioned conditions have been met. $\dot{V}(\omega)$ is safely a continuous function ($0$) thus is stable, but since $\dot{V}(\omega) = 0$ and thus can not negative, it is not asymptotically stable. This would require $V(\omega) < 0 \quad \forall \omega \epsilon D \{\omega^* = 0\}$ instead.

\section*{Part B}

We now set our $u=-K\omega$ where $k_i > 0$. We wish to show that $\omega=0$ is asymptotically stable. Thus our system of equations are now:

\[J_1 \dot{\omega}_1 = (J_2 - J_3)\omega_2 \omega_3 - k_1 \omega_1 \]
\[J_2 \dot{\omega}_2 = (J_3 - J_1)\omega_3 \omega_1 - k_2 \omega2\]
\[J_3 \dot{\omega}_3 = (J_1 - J_2)\omega_1 \omega_2 - k_3 \omega3 \]

As before, we will make the $x_1$ and $x_2$ for our system.

\[ \boldmath{x} = \begin{bmatrix}
    x_1 \\ x_2
\end{bmatrix} = \begin{bmatrix}
    x_{1_1} \\
    x_{1_2} \\ 
    x_{1_3} \\
    \\
    x_{2_1} \\
    x_{2_2} \\
    x_{2_3} 
\end{bmatrix} =
\begin{bmatrix}
    \omega_1 \\
    \omega_2 \\
    \omega_3 \\
    \\ 
    \frac{(J_2 - J_3)\omega_2 \omega_3 - k_1 \omega_1}{J_1} \\
    \frac{(J_3 - J_1)\omega_3 \omega_1 - k_2 \omega_2}{J_2} \\
    \frac{(J_1 - J_2)\omega_1 \omega_2 - k_2 \omega_3}{J_3}
\end{bmatrix}\]

We again will use the Lagrange equation to solve for our $V(\omega)$, still ignoring potential energy.

\[ V(x_1) = \sum \frac{1}{2} J_i x_{1_i}^2 \]
\[ V(x_1) = \frac{1}{2} J_1 x_{1_1}^2 + \frac{1}{2} J_2 x_{1_2}^2 + \frac{1}{2} J_3 x_{1_3}^2 \]

Our $\dot{V}(x_1)$ value does not change, and still results in $ \dot{V}(x_1) = J_1 x_{1_1} x_{2_1} + J_2 x_{1_2} x_{2_2} + J_3 x_{1_3} x_{2_3}$. Our $x_2$ values are different, however. The resulting function when expanding back to utilize $\omega$:

\[
    \dot{V}(\omega) = J_1  \omega_1 \frac{(J_2 - J_3)\omega_2 \omega_3 - k_1 \omega_1}{J_1} + J_2 \omega_2 \frac{(J_3 - J_1)\omega_3 \omega_1 - k_2 \omega_2}{J_2} + J_3 \omega3 \frac{(J_1 - J_2)\omega_1 \omega_2 - k_2 \omega_3}{J_3}
\]

\[
    \dot{V}(\omega) = (J_2-J_3)\omega_1 \omega_2 \omega_3 - k_1 \omega_1^2 + (J_3-J_1)\omega_1 \omega_2 \omega_3 - k_2 \omega_2^2 + (J_1-J_2)\omega_1 \omega_2 \omega_3 - k_3 \omega_3^2
\]

\[
    \dot{V}(\omega) = J_2 \omega_1 \omega_2 \omega_3 - J_3 \omega_1 \omega_2 \omega_3 - k_1 \omega1^2 + J_3 \omega_1 \omega_2 \omega_3 - J_1 \omega_1 \omega_2 \omega_3 - k_2 \omega_2^2 + J_1 \omega_1 \omega_2 \omega_3 - J_2 \omega_1 \omega_2 \omega_3 - k_3 \omega_3^2
\]

\[
    \dot{V}(\omega) = -k_1 \omega_1^2 - k_2 \omega_2^2 - k_3 \omega_3^2
\]

With these equations in hand, we will again explore the three conditions listed in \textbf{Part A}, again with $\omega^*=0$.

\[
    V(\omega^*) = V(0) = \frac{1}{2} J_1 0^2 + \frac{1}{2} J_2 0^2 + \frac{1}{2} J_3 0^2 = 0
\]

We see saw no changes in our $V(\omega)$ equation, so it is of no surprise that that $V(\omega^*) = 0$. Again, each $\omega$ value is squared, resulting in a positive or 0 outcome for all values This covers our first two conditions.

We now look at the derivative of $V(\omega)$, $\dot{V}(\omega)$, noting again that $k_i > 0$. We can see that due to the squaring of each $\omega$ value, that irregardless of the value of $\omega$, it will be positive. The resulting negative in front of a positive $k_i$ results in a negative value for all values of $\omega$ save $0$. Thus $\dot{V}(\omega) < 0 \quad \forall \omega \epsilon D \backslash \{\omega^*\}$, and $\dot{V}(\omega^*)=0$. This means that the system is not only stable, but asymptotically stable.

\end{document}
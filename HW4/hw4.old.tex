\documentclass{article}
\usepackage[utf8]{inputenc}
\usepackage[margin = 0.8in]{geometry}
\usepackage{graphicx}
\usepackage{amsmath, amssymb}
\usepackage{subcaption}
\usepackage{multirow}
\usepackage{mathtools}
\usepackage{float}
\setcounter{MaxMatrixCols}{20}


\title{RBE502 - Homework Set 4}
\author{Keith Chester}
\date{Due date: September ??, 2021}

\begin{document}
\maketitle

\section*{Introduction}
In this homework set we are looking at the dynamic system of a quadcopter.

The quadrotor operates with a body frame located at its center of mass (in the center of the quadcopter, equidistant and within the same plane as its rotors) $\boldsymbol{c}$. We have a fixed reference frame $\boldsymbol{e}$ serving as our intertial frame.

The external forces and moments on the system are represented by $\boldsymbol{r}$ and $\boldsymbol{n}$, where $\boldsymbol{r} = r_1 c_1 + r_2 c_2 + r_3 c_3$ and $\boldsymbol{n} = n_1 c_1 + n_2 c_2 + n_3 c_3$ directly applied to the center of mass. We are assuming that the torque of the rotor is proportionaly related to the input thrust via the constant $\sigma>0$, for $\tau_i = \sigma u_i$. We wll be utilizing $\boldsymbol{I}$ as our inertial matrix, where:

\begin{equation}
    \boldsymbol{I} = \begin{bmatrix}
        I_x & 0 & 0 \\
        0 & I_y & 0 \\
        0 & 0 & I_z \\
    \end{bmatrix}
\end{equation}

We have also had established for us that there exists a rotation matrix - $R_{C/E}$ - which rotates from frame $C$ to frame $E$. This rotaton matrix is the result of a Euler angle $z-y-x$ rotation along $\phi$,  $\theta$, and $\psi$, respectively. This rotation matrix is such that:

\begin{equation}
    R_{C/E} =
    \begin{bmatrix}
        \cos \left(\psi \right)\,\cos \left(\theta \right) & \cos \left(\psi \right)\,\sin \left(\phi \right)\,\sin \left(\theta \right)-\cos \left(\phi \right)\,\sin \left(\psi \right) & \sin \left(\phi \right)\,\sin \left(\psi \right)+\cos \left(\phi \right)\,\cos \left(\psi \right)\,\sin \left(\theta \right)\\
        \cos \left(\theta \right)\,\sin \left(\psi \right) & \cos \left(\phi \right)\,\cos \left(\psi \right)+\sin \left(\phi \right)\,\sin \left(\psi \right)\,\sin \left(\theta \right) & \cos \left(\phi \right)\,\sin \left(\psi \right)\,\sin \left(\theta \right)-\cos \left(\psi \right)\,\sin \left(\phi \right)\\
        -\sin \left(\theta \right) & \cos \left(\theta \right)\,\sin \left(\phi \right) & \cos \left(\phi \right)\,\cos \left(\theta \right)
    \end{bmatrix}
\end{equation}


\section*{System Analysis}

For our problem set, we are proivded that the equations that descirbe our system can be derived via:

\begin{equation}
    \dot{x} = \boldsymbol{v}
\end{equation}
\begin{equation}
    \dot{\alpha} = T^{-1}\omega
\end{equation}
\begin{equation}
    \dot{v} = -g \boldsymbol{e}_3 + \frac{1}{m}R_{C/E}(u_1+u2+u3+u4)\boldsymbol{c_3}+R_{C/E}\boldsymbol{r}
\end{equation}
\begin{equation}
    \dot{\omega} = \boldsymbol{I}^{-1} ((u_2 - u_4) l c_1 + (u_3-u_1) l c_2 + (u_1 - u_2 + u_3 -u_4) \sigma c_3 + \boldsymbol{n} - \boldsymbol{\omega} \times \boldsymbol{I} \boldsymbol{\omega})
\end{equation}

\begin{equation}
    T^{-1} = \begin{bmatrix}
        1 & \sin{\phi} \tan{\theta} & \cos{\phi}\tan{theta} \\
        0 & \cos{\phi} & -\sin{\phi} \\
        0 & \frac{\sin{\phi}}{\cos{\theta}} & \frac{\cos{\phi}}{\theta}
    \end{bmatrix}
\end{equation}


When we expand upon these functions, we find ourselves with a several matricies that make up a system of equations. From these expanded results we can derive the $\boldsymbol{A}$ and $\boldsymbol{B}$ matricies.

\begin{equation}
    \dot{\boldsymbol{x}} =
    \begin{bmatrix}
        v_1 \\
        v_2 \\
        v_3 \\
    \end{bmatrix}
\end{equation}

\begin{equation}
    \dot{\boldsymbol{\alpha}} =
    \begin{bmatrix}
        \omega_1 +\omega_3 \,\cos \left(\phi \right)\,\tan \left(\theta \right)+\omega_2 \,\sin \left(\phi \right)\,\tan \left(\theta \right)\\
        \omega_2 \,\cos \left(\phi \right)-\omega_3 \,\sin \left(\phi \right)\\
        \frac{\omega_3 \,\cos \left(\phi \right)}{\cos \left(\theta \right)}+\frac{\omega_2 \,\sin \left(\phi \right)}{\cos \left(\theta \right)}
    \end{bmatrix}
\end{equation}

\begin{equation}
    \dot{\boldsymbol{v}} = 
    \begin{bmatrix}
        \frac{{\left(\sin \left(\phi \right)\,\sin \left(\psi \right)+\cos \left(\phi \right)\,\cos \left(\psi \right)\,\sin \left(\theta \right)\right)}\,{\left(u_1 +u_2 +u_3 +u_4 \right)}}{m}\\
        -\frac{{\left(\cos \left(\psi \right)\,\sin \left(\phi \right)-\cos \left(\phi \right)\,\sin \left(\psi \right)\,\sin \left(\theta \right)\right)}\,{\left(u_1 +u_2 +u_3 +u_4 \right)}}{m}\\
        \frac{\cos \left(\phi \right)\,\cos \left(\theta \right)\,{\left(u_1 +u_2 +u_3 +u_4 \right)}}{m}-g
    \end{bmatrix}
\end{equation}

\begin{equation}
    \dot{\boldsymbol{\omega}} = 
    \begin{bmatrix}
        \frac{l\,{\left(u_2 -u_4 \right)}+\mathrm{Iy}\,\omega_2 \,\omega_3 -\mathrm{Iz}\,\omega_2 \,\omega_3 }{\mathrm{Ix}}\\
        -\frac{l\,{\left(u_1 -u_3 \right)}+\mathrm{Ix}\,\omega_1 \,\omega_3 -\mathrm{Iz}\,\omega_1 \,\omega_3 }{\mathrm{Iy}}\\
        \frac{\sigma \,{\left(u_1 -u_2 +u_3 -u_4 \right)}+\mathrm{Ix}\,\omega_1 \,\omega_2 -\mathrm{Iy}\,\omega_1 \,\omega_2 }{\mathrm{Iz}}
    \end{bmatrix}
\end{equation}

Now that we have these system equations, we will place them into the state equation form of $\dot{x} = Ax + Bu$. In this equation, $u$ is a vector of the input variables, and $x$ is a vector of our state variables.

\begin{equation}
    \begin{aligned}
        \begin{bmatrix}
            \dot{x}_1 \\ \dot{x}_2 \\ \dot{x}_3 \\
            \dot{\phi} \\ \dot{\theta} \\ \dot{\psi} \\
            \dot{v}_1 \\ \dot{v}_2 \\ \dot{v}_3 \\
            \dot{\omega}_1 \\ \dot{\omega}_2 \\ \dot{\omega}_3 \\
        \end{bmatrix}
        = 
        \begin{bmatrix}
            % x
            0 & 0 & 0 & 0 & 0 & 0 & 1 & 0 & 0 & 0 & 0 & 0 \\
            0 & 0 & 0 & 0 & 0 & 0 & 0 & 1 & 0 & 0 & 0 & 0 \\
            0 & 0 & 0 & 0 & 0 & 0 & 0 & 0 & 1 & 0 & 0 & 0 \\
            % alpha
            0 & 0 & 0 & 0 & 0 & 0 & 0 & 0 & 0 & 1 & \sin{\phi}\tan{\theta} & \cos{\phi}\tan{\theta} \\
            0 & 0 & 0 & 0 & 0 & 0 & 0 & 0 & 0 & 0 & \cos{\phi} & -\sin{\phi} \\
            0 & 0 & 0 & 0 & 0 & 0 & 0 & 0 & 0 & 0 & \frac{\cos(\phi)}{\sin(\theta)} & \frac{\cos(\phi)}{\cos(\theta)} \\
            % v
            0 & 0 & 0 & 0 & 0 & 0 & 0 & 0 & 0 & 0 & 0 & 0 \\
            0 & 0 & 0 & 0 & 0 & 0 & 0 & 0 & 0 & 0 & 0 & 0 \\
            0 & 0 & 0 & 0 & 0 & 0 & 0 & 0 & 0 & 0 & 0 & 0 \\
            % omega
            0 & 0 & 0 & 0 & 0 & 0 & 0 & 0 & 0 & 0 & \frac{I_y \omega_3}{I_x} & -\frac{I_z \omega_2}{I_x} \\
            0 & 0 & 0 & 0 & 0 & 0 & 0 & 0 & 0 & -\frac{I_x \omega_3}{I_y} & 0 & \frac{I_z \omega_1}{I_y} \\
            0 & 0 & 0 & 0 & 0 & 0 & 0 & 0 & 0 & \frac{I_y \omega_2}{I_z} & -\frac{I_y \omega_1}{I_z} & 0 \\
        \end{bmatrix}
        \begin{bmatrix}
            x_1 \\ x_2 \\ x_3 \\ \phi \\ \theta \\ \psi \\
            v_1 \\ v_2 \\ v_3 \\ \omega_1 \\ \omega_2 \\ \omega_3
        \end{bmatrix}
        + \\
        \begin{bmatrix}
            % x
            0 & 0 & 0 & 0 \\
            0 & 0 & 0 & 0 \\
            0 & 0 & 0 & 0 \\
            % alpha
            0 & 0 & 0 & 0 \\
            0 & 0 & 0 & 0 \\
            0 & 0 & 0 & 0 \\
            % v
            \frac{\sin{\phi}\sin{\psi}+\cos{\phi}\cos{\psi}\sin{\theta}}{m} & \frac{\sin{\phi}\sin{\psi}+\cos{\phi}\cos{\psi}\sin{\theta}}{m} & \frac{\sin{\phi}\sin{\psi}+\cos{\phi}\cos{\psi}\sin{\theta}}{m} & \frac{\sin{\phi}\sin{\psi}+\cos{\phi}\cos{\psi}\sin{\theta}}{m} \\
            -\frac{\cos{\psi}\sin{\phi}-\cos{\phi}\sin{\psi}\sin{\theta}}{m} & -\frac{\cos{\psi}\sin{\phi}-\cos{\phi}\sin{\psi}\sin{\theta}}{m} & -\frac{\cos{\psi}\sin{\phi}-\cos{\phi}\sin{\psi}\sin{\theta}}{m} & -\frac{\cos{\psi}\sin{\phi}-\cos{\phi}\sin{\psi}\sin{\theta}}{m} \\
            \frac{\cos{\phi}\cos{\theta}}{m}-\frac{g}{4u_1} & \frac{\cos{\phi}\cos{\theta}}{m}-\frac{g}{4u_2} & \frac{\cos{\phi}\cos{\theta}}{m}-\frac{g}{4u_3} & \frac{\cos{\phi}\cos{\theta}}{m}-\frac{g}{4u_4} \\
            % omega
            0 & \frac{l}{I_x} & 0 & -\frac{l}{I_x} \\
            -\frac{l}{I_y} & 0 & \frac{l}{I_y} & 0 \\
            \frac{\sigma}{I_z} & -\frac{\sigma}{I_z} & \frac{\sigma}{I_z} & -\frac{\sigma}{I_z} \\
        \end{bmatrix}
        \begin{bmatrix}
            u_1 \\ u_2 \\ u_3 \\ u_4
        \end{bmatrix}
    \end{aligned}
\end{equation}

Now that we have gotten our system into the form of $\dot{x} = Ax + Bu$, we can move forward with the questions posed by the problem set.

\section*{Problem 1 - Part A}

In this part of the problem, we are considering $ \boldsymbol{z}_0 = \begin{bmatrix}
    x_1 & x_2 & x_3 & 0 & 0 & 0 & 0 & 0 & 0 & 0 & 0 & 0
\end{bmatrix}^T$ and $ \boldsymbol{u}_0 = \begin{bmatrix}
    1 & 1 & 1 & 1
\end{bmatrix}^T \ \frac{mg}{4}$. We wish to show that $(z, u) = (z_0, u_0)$ is an equilibrium point of this quadrotor system.

First, let us look at the resulting derivatives of our system given $\boldsymbol{z}_0$ and $\boldsymbol{u}_0$.

\begin{equation}
    \begin{aligned}
        \begin{bmatrix}
            \dot{x}_1 \\ \dot{x}_2 \\ \dot{x}_3 \\
            \dot{\phi} \\ \dot{\theta} \\ \dot{\psi} \\
            \dot{v}_1 \\ \dot{v}_2 \\ \dot{v}_3 \\
            \dot{\omega}_1 \\ \dot{\omega}_2 \\ \dot{\omega}_3 \\
        \end{bmatrix}
        = 
        \begin{bmatrix}
            % x
            0 & 0 & 0 & 0 & 0 & 0 & 1 & 0 & 0 & 0 & 0 & 0 \\
            0 & 0 & 0 & 0 & 0 & 0 & 0 & 1 & 0 & 0 & 0 & 0 \\
            0 & 0 & 0 & 0 & 0 & 0 & 0 & 0 & 1 & 0 & 0 & 0 \\
            % alpha
            0 & 0 & 0 & 0 & 0 & 0 & 0 & 0 & 0 & 1 & \sin{\phi}\tan{\theta} & \cos{\phi}\tan{\theta} \\
            0 & 0 & 0 & 0 & 0 & 0 & 0 & 0 & 0 & 0 & \cos{\phi} & -\sin{\phi} \\
            0 & 0 & 0 & 0 & 0 & 0 & 0 & 0 & 0 & 0 & \frac{\cos(\phi)}{\sin(\theta)} & \frac{\cos(\phi)}{\cos(\theta)} \\
            % v
            0 & 0 & 0 & 0 & 0 & 0 & 0 & 0 & 0 & 0 & 0 & 0 \\
            0 & 0 & 0 & 0 & 0 & 0 & 0 & 0 & 0 & 0 & 0 & 0 \\
            0 & 0 & 0 & 0 & 0 & 0 & 0 & 0 & 0 & 0 & 0 & 0 \\
            % omega
            0 & 0 & 0 & 0 & 0 & 0 & 0 & 0 & 0 & 0 & \frac{I_y \omega_3}{I_x} & -\frac{I_z \omega_2}{I_x} \\
            0 & 0 & 0 & 0 & 0 & 0 & 0 & 0 & 0 & -\frac{I_x \omega_3}{I_y} & 0 & \frac{I_z \omega_1}{I_y} \\
            0 & 0 & 0 & 0 & 0 & 0 & 0 & 0 & 0 & \frac{I_y \omega_2}{I_z} & -\frac{I_y \omega_1}{I_z} & 0 \\
        \end{bmatrix}
        \begin{bmatrix}
            x_1 \\ x_2 \\ x_3 \\ 0 \\ 0 \\ 0 \\
            0 \\ 0 \\ 0 \\ 0 \\ 0 \\ 0
        \end{bmatrix}
        + \\
        \begin{bmatrix}
            % x
            0 & 0 & 0 & 0 \\
            0 & 0 & 0 & 0 \\
            0 & 0 & 0 & 0 \\
            % alpha
            0 & 0 & 0 & 0 \\
            0 & 0 & 0 & 0 \\
            0 & 0 & 0 & 0 \\
            % v
            \frac{\sin{\phi}\sin{\psi}+\cos{\phi}\cos{\psi}\sin{\theta}}{m} & \frac{\sin{\phi}\sin{\psi}+\cos{\phi}\cos{\psi}\sin{\theta}}{m} & \frac{\sin{\phi}\sin{\psi}+\cos{\phi}\cos{\psi}\sin{\theta}}{m} & \frac{\sin{\phi}\sin{\psi}+\cos{\phi}\cos{\psi}\sin{\theta}}{m} \\
            -\frac{\cos{\psi}\sin{\phi}-\cos{\phi}\sin{\psi}\sin{\theta}}{m} & -\frac{\cos{\psi}\sin{\phi}-\cos{\phi}\sin{\psi}\sin{\theta}}{m} & -\frac{\cos{\psi}\sin{\phi}-\cos{\phi}\sin{\psi}\sin{\theta}}{m} & -\frac{\cos{\psi}\sin{\phi}-\cos{\phi}\sin{\psi}\sin{\theta}}{m} \\
            \frac{\cos{\phi}\cos{\theta}}{m}-\frac{g}{4u_1} & \frac{\cos{\phi}\cos{\theta}}{m}-\frac{g}{4u_2} & \frac{\cos{\phi}\cos{\theta}}{m}-\frac{g}{4u_3} & \frac{\cos{\phi}\cos{\theta}}{m}-\frac{g}{4u_4} \\
            % omega
            0 & \frac{l}{I_x} & 0 & -\frac{l}{I_x} \\
            -\frac{l}{I_y} & 0 & \frac{l}{I_y} & 0 \\
            \frac{\sigma}{I_z} & -\frac{\sigma}{I_z} & \frac{\sigma}{I_z} & -\frac{\sigma}{I_z} \\
        \end{bmatrix}
        \begin{bmatrix}
            \frac{mg}{4} \\ \frac{mg}{4} \\ \frac{mg}{4} \\ \frac{mg}{4}
        \end{bmatrix}
    \end{aligned}
\end{equation}

\begin{equation}
    \begin{aligned}
        \begin{bmatrix}
            \dot{x}_1 \\ \dot{x}_2 \\ \dot{x}_3 \\
            \dot{\phi} \\ \dot{\theta} \\ \dot{\psi} \\
            \dot{v}_1 \\ \dot{v}_2 \\ \dot{v}_3 \\
            \dot{\omega}_1 \\ \dot{\omega}_2 \\ \dot{\omega}_3 \\
        \end{bmatrix}
        = 
        \begin{bmatrix}
            0 \\ 0 \\ 0 \\ 0 \\ 0 \\ 0 \\ 0 \\ 0 \\ 0 \\ 0 \\ 0 \\ 0
        \end{bmatrix}
        +
        \begin{bmatrix}
            0 \\ 0 \\ 0 \\ 0 \\ 0 \\ 0 \\
            g\,{\left(\sin \left(0 \right)\,\sin \left(0 \right)+\cos \left(0 \right)\,\cos \left(0 \right)\,\sin \left(0 \right)\right)}\\
            -g\,{\left(\cos \left(0 \right)\,\sin \left(0 \right)-\cos \left(0 \right)\,\sin \left(0 \right)\,\sin \left(0 \right)\right)}\\
            -g\,-g\,\cos(0)\,\cos(0) \\
            0 \\ 0 \\ 0
        \end{bmatrix}
    \end{aligned}
\end{equation}

\begin{equation}
    \begin{aligned}
        \begin{bmatrix}
            \dot{x}_1 \\ \dot{x}_2 \\ \dot{x}_3 \\
            \dot{\phi} \\ \dot{\theta} \\ \dot{\psi} \\
            \dot{v}_1 \\ \dot{v}_2 \\ \dot{v}_3 \\
            \dot{\omega}_1 \\ \dot{\omega}_2 \\ \dot{\omega}_3 \\
        \end{bmatrix}
        = 
        \begin{bmatrix}
            0 \\ 0 \\ 0 \\ 0 \\ 0 \\ 0 \\ 0 \\ 0 \\ 0 \\ 0 \\ 0 \\ 0
        \end{bmatrix}
        +
        \begin{bmatrix}
            0 \\ 0 \\ 0 \\ 0 \\ 0 \\ 0 \\
            0 \\
            0 \\
            0\\
            0 \\ 0 \\ 0
        \end{bmatrix}
    \end{aligned}
\end{equation}

\begin{equation}
    \begin{aligned}
        \begin{bmatrix}
            \dot{x}_1 \\ \dot{x}_2 \\ \dot{x}_3 \\
            \dot{\phi} \\ \dot{\theta} \\ \dot{\psi} \\
            \dot{v}_1 \\ \dot{v}_2 \\ \dot{v}_3 \\
            \dot{\omega}_1 \\ \dot{\omega}_2 \\ \dot{\omega}_3 \\
        \end{bmatrix}
        = 
        \begin{bmatrix}
            0 \\ 0 \\ 0 \\ 0 \\ 0 \\ 0 \\ 0 \\ 0 \\ 0 \\ 0 \\ 0 \\ 0
        \end{bmatrix}
    \end{aligned}
\end{equation}

Since we see in the above equations that in this position with the targeted input, we have a system in a stable equilibrium. We can claim as such as there are no non-$0$ velocities or accelerations at this point.

At any given location in space with no external forces, the drone can hover when perfectly level - ie $\phi=\theta=\psi=0$ - with no acceleration and orientation change if the resulting thrust is $\frac{mg}{4}$ - or, each motor providing resulting thrust of one fourth its weight to cancel the force of gravity.

\section*{Problem 1 - Part B}

We now wish to find the linear approximation of the quadrotor system using the vacinity of the aforementioned equilibrium point $(\boldsymbol{z_0}, \boldsymbol{u}_0)$.

To do this, we will be denoting $z_d$ as our desired $z$, and $u_d$ as our $desired$ u. We will note then that $e := z_d - z$ and $w := u_d - u$. Using Taylor Serie's approximation we can then create a linear approximation.

\begin{equation}
    f(z, u) \approx f(z_d, u_d) + \frac{\partial f}{\partial z} \Big|_{(z,u)=(z_d, u_d)}(z-z_d) + \frac{\partial f}{\partial u} \Big|_{(z,u)=(z_d,u_d)}(u-u_d)
\end{equation}

If we let our linearized approximation have $\boldsymbol{A} := \frac{\partial f}{\partial z} \Big|_{(z,u)=(z_d, u_d)}$ and $\boldsymbol{B} := \frac{\partial f}{\partial u} \Big|_{(z,u)=(z_d,u_d)}$, we can then write an equation for $\dot{e} = \boldsymbol{A}e + \boldsymbol{B}w$. To create our $\boldsymbol{A}$ and $\boldsymbol{B}$, we utilized the Jacobian of our $\dot{z}$, solved prior, to $z$ and $u$. Thus $\boldsymbol{A}$ is a $12x12$ matrix and $\boldsymbol{B}$ is a $12x4$ matrix. If we utilize the equilibrium point from Part A, we can then plug in $z_0$ and $u_0$, defined earlier. Thus:

\begin{equation}
    \boldsymbol{A} = \begin{bmatrix}
        \frac{\partial \dot{z}_1}{\partial{x_1}} & \frac{\partial \dot{z}_1}{\partial{x_2}} & \ldots & \frac{\partial \dot{z}_1}{\partial \omega_2} & \frac{\partial \dot{z}_1}{\partial \omega_3} \\
        \vdots & & \ddots & & \vdots \\
        \frac{\partial \dot{z}_{12}}{\partial{x_1}} & \frac{\partial \dot{z}_{12}}{\partial{x_2}} & \ldots & \frac{\partial \dot{z}_{12}}{\partial \omega_2} & \frac{\partial \dot{z}_{12}}{\partial \omega_3} \\
    \end{bmatrix}
\end{equation}

\begin{equation}
    \boldsymbol{B} = \begin{bmatrix}
        \frac{\partial \dot{z}_1}{\partial{u_1}} & \frac{\partial \dot{z}_1}{\partial{u_2}} & \frac{\partial \dot{z}_0}{\partial u_3} & \frac{\partial \dot{z}_0}{\partial u_4} \\
        \vdots & \vdots & \vdots & \vdots \\
        \frac{\partial \dot{z}_{12}}{\partial{u_1}} & \frac{\partial \dot{z}_{12}}{\partial{u_2}} & \frac{\partial \dot{z}_{12}}{\partial u_3} & \frac{\partial \dot{z}_{12}}{\partial u_4}
    \end{bmatrix}
\end{equation}

The $\boldsymbol{A}$ matrix is large, but can be simplified in writing by noting that, for this system it can be divided into four quadrants. Two exist as $0_6$, which represents a $6x6$ all-zero matrix. Another, $I_6$, represents a $6x6$ identity matrix. Finally, our bottom corner which we'll label as $A_{21}$, is a matrix we'll specifically define below.
\begin{equation}
    A_{21} = \begin{bmatrix}
        0 & 0 & 0 & 0 & g & 0 \\
        0 & 0 & 0 & -g & 0 & 0 \\
        0 & 0 & 0 & 0 & 0 & 0 \\
        0 & 0 & 0 & 0 & 0 & 0 \\
        0 & 0 & 0 & 0 & 0 & 0 \\
        0 & 0 & 0 & 0 & 0 & 0 \\
    \end{bmatrix}
\end{equation}
\begin{equation}
    \boldsymbol{A} = \begin{bmatrix}
        0_6 & I_6 \\
        A_{21} & 0_6
    \end{bmatrix}
\end{equation}

Our $\boldsymbol{B}$ matrix is defined below:

\begin{equation}
    \boldsymbol{B} =
    \begin{bmatrix}
        0 & 0 & 0 & 0\\
        0 & 0 & 0 & 0\\
        0 & 0 & 0 & 0\\
        0 & 0 & 0 & 0\\
        0 & 0 & 0 & 0\\
        0 & 0 & 0 & 0\\
        0 & 0 & 0 & 0\\
        0 & 0 & 0 & 0\\
        \frac{1}{m} & \frac{1}{m} & \frac{1}{m} & \frac{1}{m}\\
        0 & \frac{l}{\mathrm{Ix}} & 0 & -\frac{l}{\mathrm{Ix}}\\
        -\frac{l}{\mathrm{Iy}} & 0 & \frac{l}{\mathrm{Iy}} & 0\\
        \frac{\sigma }{\mathrm{Iz}} & -\frac{\sigma }{\mathrm{Iz}} & \frac{\sigma }{\mathrm{Iz}} & -\frac{\sigma }{\mathrm{Iz}}
        \end{bmatrix}
\end{equation}

We thus have a linearized model for the system with $\dot{e} = \boldsymbol{A}e + \boldsymbol{B}w$.

\section*{Problem 1 - Part C}

Now we are tasked with determining if our linear system is controllable. To do this, we must check to see if matrix $\boldsymbol{C}$ has full rank. We define $\boldsymbol{C}$ as:

\begin{equation}
    \boldsymbol{C} = \begin{bmatrix}
        \boldsymbol{A}^0\boldsymbol{B} & \boldsymbol{A}^1\boldsymbol{B} * \boldsymbol{A}^2\boldsymbol{B} & \ldots & \boldsymbol{A}^{n-1}\boldsymbol{B}
    \end{bmatrix}
\end{equation}

...with $n=12$ for each value within our $\dot{z}$. When we look at the rank of this matrix, we see that it is full ranked - $rank(\boldsymbol{C}) = 12$. This tells us that the linearized system is indeed controllable.


\end{document}